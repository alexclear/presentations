\documentclass[xetex,18pt,aspectratio=43]{beamer}

\usepackage{caption}
\usepackage[percent]{overpic}
\usepackage{xecyr}
\usepackage{xunicode}
\usepackage[absolute,overlay]{textpos}
\usepackage{fontspec}
\usepackage{calc}
\usepackage{multicol}
\usepackage{hyperref}
\usepackage{setspace}
\usepackage{tikz}
\usepackage{csquotes}
\usepackage[export]{adjustbox}
%\usepackage[texcoord,grid,gridunit=mm,gridcolor=red!10,subgridcolor=green!10]{eso-pic}
\defaultfontfeatures{Ligatures=TeX}
\setmainfont{Trebuchet MS}
\usepackage{polyglossia}
\setdefaultlanguage[spelling=modern]{russian}
\newfontfamily{\cyrillicfont}{Trebuchet MS}
\newfontfamily{\cyrillicfontsf}{Trebuchet MS}
\newfontfamily{\cyrillicfonttt}{Trebuchet MS}
\newfontface\lserif{Microsoft Sans Serif}

\newcommand\Bigfont{\fontsize{22}{22}\selectfont}
\newcommand\Authorfont{\fontsize{17}{17}\selectfont}
\newcommand\Orgfont{\fontsize{13}{13}\selectfont}

\mode<presentation>
{
  %\usetheme{Boadilla}      % or try Darmstadt, Madrid, Warsaw, ...
  \usecolortheme{default} % or try albatross, beaver, crane, ...
  %\usefonttheme{default}  % or try serif, structurebold, ...
  \setbeamertemplate{navigation symbols}{}
  \setbeamertemplate{caption}[numbered]
  \setbeamertemplate{itemize items}[circle]
  \setbeamerfont{title}{series=\bfseries,parent=structure}
  \setbeamerfont{frametitle}{size=\huge}
} 

\makeatother
\setbeamertemplate{footline}
{
  \leavevmode%
  \hbox{%
  \begin{beamercolorbox}[wd=.35\paperwidth,ht=2.5ex,dp=1ex,center]{author in head/foot}%
    \usebeamerfont{author in head/foot}\insertshortauthor
  \end{beamercolorbox}%
  \begin{beamercolorbox}[wd=.65\paperwidth,ht=2.5ex,dp=1ex,center]{title in head/foot}%
    \usebeamerfont{title in head/foot}\insertshorttitle\hfill
    \insertframenumber{} / \inserttotalframenumber\hspace*{-8ex}
  \end{beamercolorbox}}%
  \vskip0pt%
}
\makeatletter

\title[Современные тенденции в разработке ПО]{}
\author[Александр Чистяков, Git in Sky]{}
\date{}

\begin{document}

{ % all template changes are local to this group.
    \setbeamertemplate{navigation symbols}{}
    %\setbeamertemplate{background}[grid][step=10]
    \setbeamertemplate{background}{\includegraphics[width=\paperwidth,height=\paperheight,keepaspectratio]{img/firstslide.png}}
    \begin{frame}[plain]
      \begin{textblock*}{\framewidth}(0.95cm,3.7cm) % {block width} (coords)
        \Bigfont
          \begin{center}
          Современные тенденции в разработке ПО
          \end{center}
      \end{textblock*}
      \begin{textblock*}{\framewidth}(0.95cm,6.7cm) % {block width} (coords)
        \Authorfont
          \begin{center}
          Александр Чистяков
          \end{center}
      \end{textblock*}
      \begin{textblock*}{\framewidth}(0.95cm,7.8cm) % {block width} (coords)
        \Orgfont
          \begin{center}
          Git in Sky
          \end{center}
      \end{textblock*}
     \end{frame}
}


\begin{Large}
\begin{frame}{\ \ \ Несколько слов о себе}
\setstretch{1.2}
\begin{textblock*}{\framewidth-0.8cm}(0.5cm,1.5cm)
\begin{itemize}
  \item Главный инженер в \href{https://gitinsky.com}{\color{blue}{Git in Sky}}
  \item Преподаватель в avalon.ru
  \item Researcher @ ISST Lab, ITMO
  \item Координатор встреч DevOps-инженеров в Петербурге
  \item Пишу код
\end{itemize}
\end{textblock*}
\end{frame}

\begin{frame}{\ \ \ Слово \enquote{современные}}
\setstretch{1.2}
\begin{textblock*}{\framewidth}(0.8cm,1.5cm)
Что изображено на картинке?\\
{\small (Мы будем говорить о вещах, придуманных 30 и более лет назад)}
\begin{minipage}{\textwidth}
  \centering
  \includegraphics[height=5.5cm]{img/cassette}
\end{minipage}
\end{textblock*}
\end{frame}

\begin{frame}{\ \ \ Немного истории}
\setstretch{1.2}
\begin{textblock*}{\framewidth}(0.8cm,1.5cm)
Носитель информации 30 лет назад\\
{\small (Емкость примерно 200 килобайт)}
\begin{minipage}{\textwidth}
  \centering
  \includegraphics[height=5.5cm]{img/cassette}
\end{minipage}
\end{textblock*}
\end{frame}

\begin{frame}{\ \ \ ALGOL-60 и далее}
\setstretch{1.2}
\begin{textblock*}{\framewidth}(0.8cm,1.5cm)
  \begin{columns}[onlytextwidth,t]
    \begin{column}{0.5\textwidth}
      \centering
      \includegraphics[height=6.8cm,valign=t]{img/algorithm}
    \end{column}
    \begin{column}{0.5\textwidth}
    Структурное и процедурное программирование
    \end{column}
  \end{columns}
\end{textblock*}
\end{frame}

\begin{frame}{\ \ \ Корень всех зол (нет, не goto)}
\setstretch{1.2}
\begin{textblock*}{\framewidth-0.5cm}(1.2cm,1.5cm)
  \begin{columns}[onlytextwidth,t]
    \begin{column}{0.5\textwidth}
      Как C-программист под DSP пишет на C{\lserif\#}?\\
      {\small В C{\lserif\#} нет goto, но это не беда!}
    \end{column}
    \begin{column}{0.5\textwidth}
      \centering
      \includegraphics[height=6.8cm,valign=t]{img/spaghetti.png}
    \end{column}
  \end{columns}
\end{textblock*}
\end{frame}

\begin{frame}{\ \ \ Haskell}
\setstretch{1.2}
\begin{textblock*}{\framewidth}(0.8cm,1.5cm)
Как открыть ВАЗ 2101 без ключа?\\
{\small (Гораздо легче, чем пройти курс по Haskell*)}
\begin{minipage}{\textwidth}
  \centering
  \includegraphics[height=5.5cm]{img/lada_2101_1.jpg}
\end{minipage}
\end{textblock*}
\end{frame}

\begin{frame}{\ \ \ Выводы}
\setstretch{1.2}
\begin{textblock*}{\framewidth-0.8cm}(0.5cm,1.5cm)
\begin{itemize}
  \item Я не знаю, что будет дальше
  \item Я не знаю, какой язык лучший
  \item Поэтому писать надо на всем
  \item Но, если можете, не пишите на COBOL
  \item BTW, death can be by {\TeX} too!
\end{itemize}
\end{textblock*}
\end{frame}

\begin{frame}{\ \ \ Вопросы, пожалуйста?}
\setstretch{1.2}
\begin{textblock*}{\framewidth-0.8cm}(0.5cm,1.5cm)
\begin{itemize}
  \item ...?
  \item ...?
  \item ...?
\end{itemize}
\end{textblock*}
\end{frame}

\begin{frame}{\ \ \ That's all, folks!}
\setstretch{1.2}
\begin{textblock*}{\framewidth-0.8cm}(0.5cm,1.5cm)
\begin{itemize}
  \item \href{mailto:alex@gitinsky.com}{\color{blue}{alex@gitinsky.com}}
  \item \href{https://telegram.me/lhommequipleure}{\color{blue}{https://telegram.me/lhommequipleure}}
\end{itemize}
\end{textblock*}
\end{frame}
\end{Large}

\end{document}